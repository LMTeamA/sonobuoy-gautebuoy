\documentclass[a4paper]{article}
\usepackage[fleqn]{amsmath}
\usepackage{graphicx}
%\usepackage{times}
\usepackage[framed,numbered,autolinebreaks,useliterate]{mcode}
\usepackage{listing}
\usepackage[small,compact]{titlesec}
\usepackage[utf8]{inputenc}

\usepackage{biblatex}
\bibliography{../Documentation}

\usepackage[paper=a4paper,
            includefoot, % Uncomment to put page number above margin
            marginparwidth=30.5mm,    % Length of section titles
            marginparsep=1.5mm,       % Space between titles and text
            margin=10mm,              % 25mm margins
            includemp]{geometry}

%\setlength{\oddsidemargin}{10mm}
%\setlength{\evensidemargin}{10mm}
\usepackage{fullpage}

\usepackage{multicol}
\usepackage{caption}

\newcommand{\makeheading}[2]%
        {\hspace*{-\marginparsep minus \marginparwidth}%
         \begin{minipage}[t]{\textwidth\marginparwidth\marginparsep}%
           {\large \bfseries #1}\\{#2}\\[-0.15\baselineskip]%
                 \rule{\columnwidth}{1pt}%
         \end{minipage}}

\newlength{\figurewidth}
\setlength{\figurewidth}{500px}


\begin{document}
\makeheading{Gautebøye - Specification}{Gaute Hope
(gaute.hope@student.uib.no), 13.08.2012}

\vspace{2em}
\section*{Introduction}
The Gautebuoy..

\vspace{2em}

\begin{multicols}{2}
  \section{Overview}

  \section{Concept}

  \section{Hardware and analog system}

  \section{Digital control system}

  \section{Timing}
  The timing is based on the GPS unit time and date information as well
  as a pulse per second (PPS) signal which has an accuracy of up to 1
  $\mu$s \cite{em_406A_gps_ds}, this represents the limit on the timing. The PPS signal will
  only be present if the GPS has a sufficiently good enough signal (fix),
  which in the open air in the arctic should be no problem. If a PPS
  signal is present the status flag HAS\_SYNC is set, if time and date
  is present the status flag HAS\_TIME is set.

  \subsection{Time representation}
  The time is represented as seconds since UNIX epoch, 1970, the GPS
  time is used directly and \textbf{no} additional leap seconds are added.

  \paragraph{}When needed the time is represented as microseconds since
  UNIX epoch, i.e. for the batch reference. All operations thus require
  64 bit wide unsigned integers.

  \subsection{Determining time}
  The current second is determined using the following steps
  (implemented in buoy/gps.cpp):

  \begin{enumerate}
    \item Received time and date telegram
      \subitem Disable PPS handler
      \subitem Determine seconds since UNIX epoch (1970)
      \subitem Enable PPS handler
    \item Receive PPS signal
      \subitem Increment second
      \subitem Record output of micros(), the internal CPU time in
      microseconds, this is called the 'microdelta'.
  \end{enumerate}

  To get a reference accurate to a microsecond, apart from the drift
  of the internal CPU clock since the last PPS append the delta of a new
  call to micros(), the CPU time in microseconds, and the recorded value
  of micros() at the time of the PPS signal, to the second determined at
  the time of the PPS signal.

  \paragraph{Assumptions}
  \begin{enumerate}
    \item When time and date is fixed the next PPS is for the next
      second, otherwise the time would already also be one second
      later.
    \item There will be a good enough fix, under normal conditions, for
      the PPS signal often enough that micros() will not overflow and
      cause a backwards jump in time before a new reference has been set.
    \item There cannot be PPS pulse without a valid time.
  \end{enumerate}

  \subsection{Determining a new reference}
  Every time there is a PPS signal a
  new reference is made available with a fresh 'microdelta'. The
  continuously refreshed reference is not used before a new batch is
  started. With a batch length of 1024 samples and a sample rate of 250
  Hz it takes approximately 4 seconds before it is full. This is
  implemented in: buoy/ads1282.cpp and buoy/gps.cpp.

  \subsection{Drift}
  The drift of the CPU clock is bounded by the crystal accuracy,
  specified in ppm, as well as the number of CPU instructions around the
  causing any lag. The CPU instructions are in the order of 5-6 lines of
  C-code and given that the CPU runs at 72 MHz are ignored
  \footnote{TODO: Better ref: \cite{stm32f103rbt6_ds}}.

  \paragraph{The crystal} \textit{TODO:} Figure out which crystal Olimexino is
  using and calculate drift over: a normal reference update cycle and a
  long cycle without PPS.

  \section{Protocol}

  \section{Zero: Central logging point}


\vspace{5em}
\printbibliography
\end{multicols}

\end{document}

