\documentclass[a4paper]{article}
\usepackage[fleqn]{amsmath}
\usepackage{graphicx}
%\usepackage{times}
\usepackage[framed,numbered,autolinebreaks,useliterate]{mcode}
\usepackage{listing}
\usepackage[small,compact]{titlesec}
\usepackage[utf8]{inputenc}

\usepackage{biblatex}
\bibliography{../Documentation}

\usepackage[paper=a4paper,
            includefoot, % Uncomment to put page number above margin
            marginparwidth=30.5mm,    % Length of section titles
            marginparsep=1.5mm,       % Space between titles and text
            margin=10mm,              % 25mm margins
            includemp]{geometry}

%\setlength{\oddsidemargin}{10mm}
%\setlength{\evensidemargin}{10mm}
\usepackage{fullpage}

\usepackage{multicol}
\usepackage{caption}

\newcommand{\makeheading}[2]%
        {\hspace*{-\marginparsep minus \marginparwidth}%
         \begin{minipage}[t]{\textwidth\marginparwidth\marginparsep}%
           {\large \bfseries #1}\\{#2}\\[-0.15\baselineskip]%
                 \rule{\columnwidth}{1pt}%
         \end{minipage}}

\newlength{\figurewidth}
\setlength{\figurewidth}{500px}


\begin{document}
\makeheading{Gautebøye - Processing}{Gaute Hope
(gaute.hope@student.uib.no), 21.08.2012, Revision 1}

\vspace{2em}
\section*{Introduction}
This document describes the simple processing tools as well as some
notes for how to use the data with SEISAN or MATLAB.

\vspace{2em}

\begin{multicols}{2}
  \section{Tools}
  \subsection{dtttomseed}
  \subsection{dattodtt}
  \subsection{dttfix}
  \subsection{fakedtt}
  \subsection{mschangesource}

  \section{SEISAN}
    \subsection{Configuration}
    You have to increase the plot resolution of mulplt to something like
    40000 horizontal and around 3000 vertical, as well as update the
    frequency band of the spectral analysis to at least 125 Hz.

    \subsection{Conversion}
    Use \textit{dtttomseed} to convert to miniSEED, update the source
    file to change location and station. I have used SHZ as channel code
    for the hydrophone.

    \subsection{Test}
    Use the program \textit{fakedtt.py} to create a DTT file with a sine
    wave and try to convert and plot it to see if everything works.

    \subsection{Importing to SEISAN}
      \begin{enumerate}
        \item Create a work directory, copy the data and index files for
          the events to it.
        \item Use dtttomseed or dattomseed with the first argument being
          station name, BUO1 to BUO5 and after that a list of sequences
          of the ids of the files.
        \item A mseed file is created in the current directory.
        \item Use mscut to split the file up in hourly intervals.
        \item Create a list of the final files with dirf.
        \item Use autoreg to add the events to the database. Copy the
          WAV files and select create new IDs if they conflict with
          existing events.
        \item Use associ to merge the events with the existing events in
          the database. Delete merged events.
      \end{enumerate}
  \section{MATLAB}

%\vspace{1em}
%\printbibliography

\end{multicols}
\end{document}

