\documentclass[a4paper]{article}
\usepackage[fleqn]{amsmath}
\usepackage{graphicx}
%\usepackage{times}
\usepackage[framed,numbered,autolinebreaks,useliterate]{mcode}
\usepackage{listing}
\usepackage[small,compact]{titlesec}
\usepackage[utf8]{inputenc}

\usepackage{biblatex}
\bibliography{../Documentation}

\usepackage[paper=a4paper,
            includefoot, % Uncomment to put page number above margin
            marginparwidth=30.5mm,    % Length of section titles
            marginparsep=1.5mm,       % Space between titles and text
            margin=10mm,              % 25mm margins
            includemp]{geometry}

%\setlength{\oddsidemargin}{10mm}
%\setlength{\evensidemargin}{10mm}
\usepackage{fullpage}

\usepackage{multicol}
\usepackage{caption}

\newcommand{\makeheading}[2]%
        {\hspace*{-\marginparsep minus \marginparwidth}%
         \begin{minipage}[t]{\textwidth\marginparwidth\marginparsep}%
           {\large \bfseries #1}\\{#2}\\[-0.15\baselineskip]%
                 \rule{\columnwidth}{1pt}%
         \end{minipage}}

\newlength{\figurewidth}
\setlength{\figurewidth}{500px}


\begin{document}
\makeheading{Gautebøye - File specification}{Gaute Hope
(gaute.hope@student.uib.no), 13.08.2012, Revision 2 (30.11.2012)}

\vspace{2em}
\section*{Introduction}
This document is a specification of how data is stored locally on the
buoy and on the central logging point, versions \textbf{10 for buoy
storage (DAT)} and \textbf{3 for central storage (DTT)}.

\vspace{2em}

\begin{multicols}{2}
  \section{Overview}
  There are two file formats used: one binary
  and one ASCII format who both may fully represent the data.
  The binary format is used as local storage on the
  buoy SD-card while the ASCII format is used on the central logging point
  (Zero).

  \section{Data structure}
    \subsection{Entity}
    The data consists of:
    \begin{enumerate}
      \item Sample: Values returned from AD converter (32 bit)
      \item Timestamps
      \item Position
      \item Status (Validity of timing and position)
    \end{enumerate}

    \paragraph{} The formats are designed to uniquely represent these
    entities.

    \subsection{Structure}
    The data is organized in IDs with a number of batches, each batch
    has a reference and a set of corresponding values. The terms reference and
    batch are used interchangeably.

    \begin{tabbing}
    Store\= \+ \\
      ID 1 \=  \+ \\
        Reference 0 \= \\
        Reference 1 \= \+ \\
          Sample 0 \\
          Sample 1 \\
          \dots \\
          Sample 1023 \- \\
        Reference 2 \\
        \dots \\
        Reference 39 \- \\
      ID 2 \\
      ID 3 \\
      \dots
    \end{tabbing}
    \captionof{table}{Store structure}
    \label{tbl:store_structure}

    There is one set of data and index file for each ID, the index file
    contains information about the number of batches (references) and
    samples. The data file contains the references and the samples. The
    index acts as an header to the data file with general information
    and status. Status and information that changes with each batch or
    reference is stored along with the samples in the data file.

    \subsubsection{Index}
    The index contains:
    \begin{enumerate}
      \item Version of file format (incremented with buoy updates even
        when the format stays the same)
      \item ID
      \item Sample length (always 4 bytes, 32 bits)
      \item Number of samples (batch size $\cdot$ max references)
      \item Number of samples per reference (batch size, 1024)
      \item Number of references (40)
      \item Flag indicating whether there was an error with writing
        samples fast enough to card
    \end{enumerate}

    \subsubsection{Reference: Timing, position and status}
    Each reference determines the position and status as well as a
    time reference for its batch of samples. That means there is not
    recorded a position and status, as well as timestamp, for each
    sample - but for a predefined number of samples. The
    time of each samples are calculated from the reference and the
    sample rate. The batch size is chosen based on the desired resolution of
    timestamps and position fixes and limitations of memory of the CPU
    unit, it is put to 1024 samples per batch, which
    results in little drift\footnote{TODO: Put some numbers on timing drift..} between each reference and provides
    frequent positioning at every $1024 / 250 \text{Hz} \approx 4 \text{s}$.

    \paragraph{} The time is represented as microseconds since the UNIX
    epoch (1970-1-1 00:00:00), this is incidentally the same as the high
    precision time format (hptime\_t) defined by MiniSEED \cite{libmseed_man}.

    \paragraph{Checksum}
    As the data is sampled a 32-bit checksum is calculated the moment a
    sample is read from the AD unit for each
    batch. This is stored along with the reference and is used to check
    whether the stored data has not been corrupted and is the same that
    was recorded. The checksum is the XOR between all samples in the
    batch.

    \paragraph{Entity}
    A reference has the following entities:
    \begin{enumerate}
      \item Reference number
      \item Time reference: microseconds since Unix epoch
      \item Position (longitude and latitude)
      \item Status
      \item Checksum
    \end{enumerate}

    \paragraph{}
    Both formats use this structure, on the buoys local storage it is
    stored in the root folder of the SD card and on the central it is
    stored in a folder dedicated to each buoy.

    \subsubsection{Status} The status is a 16-bit wide unsigned integer
    with bits set for different conditions, the bit is set if the
    specified function is working with a optimal status of 15 (0b1111), these are defined in buoy/gps.h: \\

    \vspace{1em}

    \begin{tabular}{|l|l|}
      \hline
      \bfseries Bit & \bfseries Flag \\ \hline
       0b0    &   NOTHING \\ \hline
       0b1    &   HAS\_TIME    \\ \hline
       0b10   &   HAS\_SYNC    \\ \hline
       0b100  &   HAS\_SYNC\_REFERENCE \\ \hline
       0b1000 &   POSITION \\ \hline
    \end{tabular}
    \captionof{table}{Status flag bit mapping}
    \label{tab:status_flags}

    \paragraph{NOTHING} No flags set.

    \paragraph{HAS\_TIME} A valid time is provided by the GPS
    unit.
    \paragraph{HAS\_SYNC} There is a PPS signal being sent and the
    timing can be determined accurately.

    \paragraph{HAS\_SYNC\_REFERENCE} The reference in use was
    determined at a time when there was a PPS signal available. This can
    remain true after both time and sync is lost and the accuracy will
    deteriorate with the drift of the CPU clock.

    \paragraph{POSITION} The GPS is delivering a valid
    position fix. This will be true when HAS\_TIME is true.

    \subsection{Sample} The sample is 32 bits wide and is an integer
    stored as two's complement. The least significant bit (LSB)
    indicates whether the full scale range was exceeded and the output
    has been clipped. If the value is at its maximum and the LSB is set
    (1) the output has been clipped, if the value is at its minimum and
    LSB is unset (0) the output has been clipped \cite{ads1282_ds}. For
    correct interpretation of the sample the LSB should be set to 0
    before attempting to read it as a 32 bit integer.

    \paragraph{} Throughout processing and storing the values are
    represented with 32 bits, even though the least significant bit may
    be cleared after having converted the values to signed integers.

  \section{Formats} Since the files are stored in DAT format then read
  either directly from the SD card or sent over the network the DTT
  version of an ID is determined entirely by the properties of its
  master DAT file. The settings for size or batch length in the DAT will
  be seen in the DTT file.

  \subsection{Binary format: DAT}
  The binary format uses the extension
  DAT for its data files and IND for its index files, the buoy chooses
  the next free ID number on start up picking the next free ID as a data
  file is filled up. This puts a limit at the maximum number of IDs
  since the FAT32 file system has a limit on the maximum number of characters.

  \paragraph{Endianness} The buoy runs on an ARM7 CPU and values are
  stored in little endian.

  \paragraph{Version} The version number, stored in the index,
  is updated whenever there is an upgrade of the buoy so that the
  data quality or integrity changes. The buoy version is not stored.
  \footnote{TODO: Mentioned in Entity: Version}

  \paragraph{Implementation}
  The implementation and latest version of the DAT file format is to be
  found in the source: buoy/store.h and buoy/store.cpp.

  \paragraph{Limit of IDs}
  The SD card is formatted using the FAT32 file
  system and the file names are limited by the 8.3 format, that leaves 8
  digits for the ID number and a maximum ID of $10^8 - 1 = 99999999$.
  With a maximum number of 40 references (batches) per data file and a
  batch size of 1024 values there will be ~164 seconds of recording in
  each file. This means that for 14 days of continuous recording 7383
  IDs are needed, in other words well below the limit
  \cite{local_data_capacity_budget}.

  \paragraph{Data file size}
  The number of references (batches) and the data file size has been
  chosen to be in this order so that in case of data corruption in one
  file not too many samples will be lost, but still to be balanced
  against the overhead of more IDs and metadata. The more specific a
  batch of samples can be addressed the easier it is to retrieve them
  with the minimum disturbance of other batches or IDs. To avoid any
  interference with the current logging a data file is not transmitted
  before it is closed, that means that there is a lag the length of the
  data file when watching the data in real time. The chosen length is a
  reasonable compromise between these.


  \subsubsection{Structure: Index}
  The index file for ID is named \textit{ID.IND} and is written when a
  data file is full, the file structure is:
  \vspace{1em}

  \begin{tabular}{|r|c|c|l|}
    \hline
    \bfseries Start & \bfseries Field & \bfseries  Type & \bfseries Size  \\
    \hline
    0   & Version             & uint16\_t & 2 \\ \hline
    2   & Id                  & uint32\_t & 4 \\ \hline
    6   & Sample length       & uint16\_t & 2 \\ \hline
    8   & No of samples       & uint32\_t & 4 \\ \hline
    12  & Batch size          & uint32\_t & 4 \\ \hline
    16  & No of references    & uint32\_t & 4 \\ \hline
    20  & SD lag flag         & uint8\_t  & 1 \\ \hline
    & \textbf{Total length}   &           & \textbf{21} \\ \hline
  \end{tabular}
  \captionof{table}{Binary index file, fields, start and size is given in bytes.}
  \label{tab:index_binary_fields}

  \subsubsection{Structure: Data file}
  The data file is made up of batches, consisting of first one reference
  then a sequence of samples. The batches are concatenated after each
  other without spacing as described in table \ref{tbl:store_structure}.
  The structure of the reference is:
  \vspace{1em}

  \begin{tabular}{|r|c|c|l|}
    \hline
    \bfseries Start   &     \bfseries Field     &   \bfseries Type
    &   \bfseries Size  \\ \hline
    0   & Padding           & 0           & 12 \\ \hline
    12  & Reference id      & uint32\_t   & 4 \\ \hline
    16  & Time (reference)  & uint64\_t   & 8 \\ \hline
    24  & Status            & uint32\_t   & 4 \\ \hline
    28  & Latitude          & char[12]    & 12 \\ \hline
    40  & Longitude         & char[12]    & 12 \\ \hline
    52  & Checksum          & uint32\_t   & 4 \\ \hline
    56  & Padding           & 0           & 12  \\ \hline
    & \textbf{Total length} &  & \textbf{68} \\ \hline
  \end{tabular}
  \captionof{table}{Binary reference, fields, start and size is given in bytes.}
  \label{tab:reference_binary_fields}

  \vspace{1em}

  The reference is padded with zeros ($3 \cdot \text{sample length} =
  12$) in both ends so that it will be easier to
  recover parts of a corrupt data file as well as making it possible to
  detect a reference when none is expected. The reference id starts at 0
  for each data file reaching up to 40 which is the number of batches
  before a data file is considered full.

  \paragraph{Sample}Each sample is an \textbf{uint32\_t} with size 4.

  \paragraph{Batch}The structure of one
  batch is then:
  \vspace{1em}

  \begin{tabular}{|r|c|c|l|}
    \hline
    \bfseries Start   &     \bfseries Field     &   \bfseries Count
    &   \bfseries Size \\ \hline
    0   & Reference & 1 & 68 \\ \hline
    68  & Samples   & 1024 & 4096 \\ \hline
    & \textbf{Total length} &  & \textbf{4164} \\ \hline
  \end{tabular}
  \captionof{table}{Binary batch, fields, start and size is given in bytes.}
  \label{tab:batch_binary_fields}

  \vspace{1em}
  Such that a data file is given as: \\
  \vspace{1em}

  \begin{tabular}{|r|c|c|l|}
    \hline
    \bfseries Start   &     \bfseries Field     &   \bfseries Count
    &   \bfseries Size  \\ \hline
    0   & Reference 0 & 1 & 68 \\ \hline
    68  & Samples   & 1024 & 4096 \\ \hline
    4164 & Reference 1 & 1 & 68 \\ \hline
    4232  & Samples   & 1024 & 4096 \\ \hline
    \dots & \dots & \dots & \dots \\ \hline
    162396 & Reference 39 & 1 & 68 \\ \hline
    162464 & Samples   & 1024 & 4096 \\ \hline
    & \textbf{Total length} &  & \textbf{166560} \\ \hline
  \end{tabular}
  \captionof{table}{Binary data file, fields, start and size is given in bytes.}
  \label{tab:data_binary_fields}


  \subsection{ASCII format: DTT}
  The ASCII format uses the extension DTT for its data file and ITT for
  its index files, they are created by the central logger (Zero) as
  they are transmitted from the buoy. A file may be incompletely
  downloaded and still be readable. The index file stores a list of all
  references and how much of it has been downloaded (chunks), normally
  a whole reference is downloaded at the time, but the logger may be
  configured to download only parts (chunks) of the batch at the time.
  All line endings are UNIX line endings '$\backslash$n'.

  \paragraph{}The logger is written in Python so the file format has
  been influenced by this. It is read directly by MATLAB scripts
  which plot and do simple analyzes.

  \paragraph{}The central logger also keeps another index file,
  \textit{indexes}, with a list of available IDs on the buoy and whether
  they are enabled (file exists and is valid on buoy).

  \paragraph{}The implementation and latest version is to be found in
  the source code: zero/data.py for index and data, and zero/index.py
  for the index of all available ids.

  \subsubsection{Structure: Index of ids}
  This is a list with one line per id of the format: \\
  \begin{center}
  \textit{ID,enabled}
  \end{center}

  \begin{enumerate}
    \item ID: integer value id
    \item enabled: 'True' or 'False', whether this ID exists on the
      buoy.
  \end{enumerate}

  \subsubsection{Structure: Index}
  There is one field per line with the following fields: \\
  \begin{enumerate}
    \item Local version (version of DTT format) (integer)
    \item Remote version (version of DAT format on buoy) (integer)
    \item ID (integer)
    \item Number of samples (integer)
    \item Number of references (integer)
    \item HasFull, 'True' or 'False', whether a complete index has been
      received.
    \item SD lag flag, 'True' or 'False', whether data could
      \textit{not} be written fast enough to card.
  \end{enumerate}

  After that there is one line for each received reference, with a comma
  separated list of the following fields: \\
  \begin{enumerate}
    \item Reference number (integer)
    \item Reference timestamp (integer), microseconds since Unix epoch.
    \item Status (integer)
    \item Latitude  (string)
    \item Longitude (string)
    \item Checksum (integer)
    \item Line (integer), specifying where this batch starts in data file.
    \item Comma separated list of finished chunks (integers, variable
      length)
  \end{enumerate}

  If the chunk size equals the batch size, there is only one chunk per
  reference. Meaning if the reference (batch) has been retrieved there
  should be one '0' in the last comma separated list. There cannot be a
  chunk size greater than the batch size and the batch size must be a
  multiple of the chunk size.

  \subsubsection{Structure: Data file}
  The data file is built up in a similar way as the binary data file,
  where all the batches follow each other, a batch consists of a
  reference followed by a sequence of samples. So that the timing of
  each sample can be determined from the previous reference.

  The reference is one line of a comma separated list of the fields: \\
  \begin{enumerate}
    \item R (the character 'R' to indicate a reference)
    \item Batch length (integer), value: 1024
    \item Reference number (integer)
    \item Reference timestamp (integer)
    \item Status (integer)
    \item Latitude (string)
    \item Longitude (string)
    \item Checksum (integer)
  \end{enumerate}

  After this follows 1024 samples (integers) one on each line. The
  references are sorted ascending, but there may be missing references
  (batches).

  %\section{Working with DTTs and DATs}
  %Refer to \textit{doc/Processing} for a description of the programs and tools
    %that work with the DAT and DTT files \cite{processing}.

\vspace{5em}
\printbibliography
\end{multicols}

\end{document}

